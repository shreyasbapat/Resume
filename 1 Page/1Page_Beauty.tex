%%%%%%%%%%%%%%%%%%%%%%%%%%%%%%%%%%%%%%%
% Deedy - One Page Two Column Resume
% LaTeX Template
% Version 1.1 (30/4/2014)
%
% Original author:
% Debarghya Das (http://debarghyadas.com)
%
% Original repository:
% https://github.com/deedydas/Deedy-Resume
%
% IMPORTANT: THIS TEMPLATE NEEDS TO BE COMPILED WITH XeLaTeX
%
% This template uses several fonts not included with Windows/Linux by
% default. If you get compilation errors saying a font is missing, find the line
% on which the font is used and either change it to a font included with your
% operating system or comment the line out to use the default font.
% 
%%%%%%%%%%%%%%%%%%%%%%%%%%%%%%%%%%%%%%
% 
% TODO:
% 1. Integrate biber/bibtex for article citation under publications.
% 2. Figure out a smoother way for the document to flow onto the next page.
% 3. Add styling information for a "Projects/Hacks" section.
% 4. Add location/address information
% 5. Merge OpenFont and MacFonts as a single sty with options.
% 
%%%%%%%%%%%%%%%%%%%%%%%%%%%%%%%%%%%%%%
%
% CHANGELOG:
% v1.1:
% 1. Fixed several compilation bugs with \renewcommand
% 2. Got Open-source fonts (Windows/Linux support)
% 3. Added Last Updated
% 4. Move Title styling into .sty
% 5. Commented .sty file.
%
%%%%%%%%%%%%%%%%%%%%%%%%%%%%%%%%%%%%%%%
%
% Known Issues:
% 1. Overflows onto second page if any column's contents are more than the
% vertical limit
% 2. Hacky space on the first bullet point on the second column.
%
%%%%%%%%%%%%%%%%%%%%%%%%%%%%%%%%%%%%%%

\documentclass[a4paper]{deedy-resume-openfont}
%\newgeometry{paperwidth=8.27in,paperheight=11.69in}

\begin{document}

%%%%%%%%%%%%%%%%%%%%%%%%%%%%%%%%%%%%%%
%
%     LAST UPDATED DATE
%
%%%%%%%%%%%%%%%%%%%%%%%%%%%%%%%%%%%%%%
\lastupdated

%%%%%%%%%%%%%%%%%%%%%%%%%%%%%%%%%%%%%%
%
%     TITLE NAME
%
%%%%%%%%%%%%%%%%%%%%%%%%%%%%%%%%%%%%%%


\namesection{Shreyas}{Bapat}{ \urlstyle{same}\url{https://shreyasb.com} \\
\href{mailto:b16145@students.iitmandi.ac.in}{b16145@students.iitmandi.ac.in} | \href{mailto:bapat.shreyas@gmail.com}{bapat.shreyas@gmail.com} | (+91) 973 621 0570
}

%%%%%%%%%%%%%%%%%%%%%%%%%%%%%%%%%%%%%%
%
%     COLUMN ONE
%
%%%%%%%%%%%%%%%%%%%%%%%%%%%%%%%%%%%%%%

\begin{minipage}[t]{0.33\textwidth} 

%%%%%%%%%%%%%%%%%%%%%%%%%%%%%%%%%%%%%%
%     EDUCATION
%%%%%%%%%%%%%%%%%%%%%%%%%%%%%%%%%%%%%%

\section{Education} 

\subsection{Indian Institute of \\Technology Mandi}
\descript{BTech in Electrical Engineering}
\location{Expected Jun 2020 | Mandi, India \\Cum. GPA: 7.51}
\sectionsep


%\descript{BS in Computer Science}
%\location{Expected May 2014 | Ithaca, NY}
%Conc. in Software Engineering \\
%College of Engineering \\
%Dean's List (All Semesters) \\
%\location{ Cum. GPA: 3.92 / 4.0 \\
%Major GPA: 3.94 / 4.0}
%\sectionsep

\subsection{Miss Hill H. Sec. School}
\descript{Class XII}
\location{Grad. March 2016 |  Gwalior, India \\ Central Board of Secondary Education \\ Percentage : 93.4\%}
\sectionsep

\subsection{Kendriya Vidyalaya No. 1}
\descript{Class X}
\location{Grad. March 2014 |  Gwalior, India \\ Central Board of Secondary Education \\ CGPA : 10}
%\sectionsep


%%%%%%%%%%%%%%%%%%%%%%%%%%%%%%%%%%%%%%
%     LINKS
%%%%%%%%%%%%%%%%%%%%%%%%%%%%%%%%%%%%%%

\section{Links} 
GitHub:// \href{https://github.com/shreyasbapat}{\custombold{shreyasbapat}} \\
(Every project is made open on GitHub)\\
LinkedIn://  \href{https://www.linkedin.com/in/shreyasbapat}{\custombold{shreyasbapat}} \\
%\sectionsep

%%%%%%%%%%%%%%%%%%%%%%%%%%%%%%%%%%%%%%
%     COURSEWORK
%%%%%%%%%%%%%%%%%%%%%%%%%%%%%%%%%%%%%%

\section{Coursework}

\subsection{Undergraduate}
Data Structures and Algorithms\\
Pattern Recognition\\
Artificial Intelligence\\
Introduction to Communicating Distributed Processes\\
Computer Organisation\\
Signals and Systems
%\sectionsep

%%%%%%%%%%%%%%%%%%%%%%%%%%%%%%%%%%%%%%
%     SKILLS
%%%%%%%%%%%%%%%%%%%%%%%%%%%%%%%%%%%%%%

\section{Skills}
\subsection{Programming}
Python \textbullet{}  C++ \textbullet{} C \textbullet{}
Flutter\\ Erlang \textbullet{} Assembly \\

\subsection{Web Development}
%\location{Intermediate:}
Flask \textbullet{}  Dash \textbullet{} CSS3 \textbullet{} HTML\\
%\location{Beginner:}
Sphinx \textbullet{} Django\\
%\sectionsep
\subsection{Tools/Markup}
%\location{Tools:}
git \textbullet{}  virtualenv \textbullet{} numba \textbullet{} Keras \\
%\location{Markup:}
\LaTeX \textbullet{} ReStrctured Text \textbullet{} YAML\\
%\sectionsep
\section{Awards}
KVPY Scholar 2016-17\\
Aryabhat Astronomy Quiz: Rank 2/4500\\
Mentor @ Astronomy Code Camp Delhi

\section{Interests}
Deep Learning\\
Data Visualisation\\
Astrodynamics

%%%%%%%%%%%%%%%%%%%%%%%%%%%%%%%%%%%%%%
%
%     COLUMN TWO
%
%%%%%%%%%%%%%%%%%%%%%%%%%%%%%%%%%%%%%%

\end{minipage} 
\hfill
\begin{minipage}[t]{0.66\textwidth} 



%%%%%%%%%%%%%%%%%%%%%%%%%%%%%%%%%%%%%%
%     EXPERIENCE
%%%%%%%%%%%%%%%%%%%%%%%%%%%%%%%%%%%%%%

\section{Work Experience}

\runsubsection{poliastro}
\descript{| Software Development Intern }
\location{May 2018 - Aug 2018 | Work From Home}
\vspace{\topsep} % Hacky fix for awkward extra vertical space
\begin{tightemize}
\item Implemented interactive 2D plotting, refactoring the plotting module to create backends and orbit simulation. Fixed hyperbolic orbits.
\item Developed module being used by scientists in ESA (European Space Agency) to simulate orbits of various objects in space.

\end{tightemize}
%\sectionsep

%\runsubsection{Google}
%\descript{| Software Engineering Intern }
%\location{May 2013 – Aug 2013 | Mountain View, CA}
%\begin{tightemize}
%\item Worked on the YouTube Captions team in primarily vanilla Javascript and Python to plan, design and develop the full stack implementation of a new framework to add and edit Automatic Speech Recognition captions.\item Created a backbone.js-like framework for the Captions editor.\item All code was reviewed, perfected, and pushed to production.\end{tightemize}
%\sectionsep

%\runsubsection{Phabricator}
%\descript{| Open Source Contributor \& Team Leader}
%\location{Jan 2013 – May 2013 | Palo Alto, CA \& Ithaca, NY}
%\begin{tightemize}
%\item Phabricator is used daily by Facebook, Dropbox, Quora, Asana and more.\item I created the Meme generator, the entire Lipsum application, ported Tokens to different apps, fixed many bugs and more in PHP and Shell.\item Led a team from MIT, Cornell, IC London and UHelsinki for the project.\end{tightemize}
%\sectionsep

\section{Publications}

\runsubsection{Projecthiko 1.O - The Voice and Internet Enabled Smart Home}
\descript{| Co-Author}
\location{June 2017 | IJETSR ISSN: 2394-3386}
Cost Reduction in home automation. Used flask for handling backend. Implemented Speech Recognition.  
%\sectionsep

%%%%%%%%%%%%%%%%%%%%%%%%%%%%%%%%%%%%%%
%     RESEARCH
%%%%%%%%%%%%%%%%%%%%%%%%%%%%%%%%%%%%%%

\section{Projects}
\runsubsection{VLBI Image Reconstruction}
\descript{| Undergraduate Research}
\location{June 2018 – Present | Mandi, India}
Reconstruction of Radio Sprectrum Data taken by Event Horizon Telescope using Deep Learning. Currently using Autoencoders. 
Technologies Used: Python, Keras, Tensorflow, 
\sectionsep

\runsubsection{Ego-Nonego Video Detection}
\descript{| Undergraduate Research}
\location{Feb 2018 - May 2018 | Mandi, India}
%\vspace{\topsep} % Hacky fix for awkward extra vertical space
\begin{tightemize}
\item Video Classification on the basis of position of camera. \item Implemented a Autoencoder to create Optical Flows by taking video frames. Used ResNet50 for classification.
\item Technologies Used: Python, matplotlib, MATLAB, Keras, Tensorflow
\end{tightemize}
\sectionsep

\runsubsection{Fabric Detection}
\descript{| Undergraduate Research}
\location{Jan 2018 - Mar 2018| Mandi, India}
\begin{tightemize}
\item Implemented Transfer Learning to train an encoder to reduce dimensions of microscopic fabric images. \item Used VGG network to classify the bottlenecks. Used tSNE to cluster the various classes. Technologies Used: Python, Keras
\end{tightemize}
\sectionsep

%\runsubsection{Poliastro/poliastro}
%\descript{| Open Source Contributor}
%\location{Jan 2018 – Present | Mandi, India}
%Contributed to a Open Source Python library, \textbf{\href{http://github.com/poliastro/poliastro}{Poliastro}} - a library for orbital mechanics and related computations. 
%\sectionsep

\runsubsection{Astrool/astrool}
\descript{| Lead Developer}
\location{Jan 2018 – Present | Mandi, India}
\begin{tightemize}
\item Author of a Python Library, \textbf{\href{http://github.com/astrool/astrool}{Astrool}} - a library for computations related to positional astronomy and map generations. 
\item Published as a Pypi Package.
\end{tightemize}
\sectionsep

\runsubsection{Ayushman Bhava}
\descript{| Design Practicum Project}
\location{Frb 2018 – May 2018 | Mandi, India}
Created a smart medical vending machine with facility to contact doctor through video call. Applied Deep learning for false currency detection and classification.
\sectionsep

\runsubsection{Exoplanet Detection}
\descript{| Developer}
\location{Jan 2018 | Chennai, India}
Used K-Neighbour Classifier to classify if a given time series light data is from a star having exoplanets or not. Implemented SMOTE. 
Technologies Used: Python, matplotlib, pandas, scipy
%\sectionsep








%%%%%%%%%%%%%%%%%%%%%%%%%%%%%%%%%%%%%%
%     AWARDS
%%%%%%%%%%%%%%%%%%%%%%%%%%%%%%%%%%%%%%

%\section{Awards} 
%\begin{tabular}{rll}
%2013	     & 5th/7500  & Aryabhat Astronomy Quiz\\
%2014	     & 2nd/8200  & Aryabhat Astronomy Quiz\\
%2017	     & 1\textsuperscript{st}  & Technex-Exploring the Interstellar, IIT BHU\\
%2017	     & 4\textsuperscript{th}  & Eyes on the Sky - Inter IIT Tech Meet, IIT Kanpur\\
%2017     & Presented a paper & International Conference ICETSMI at IETE, Delhi  \\
%2018     & 5\textsuperscript{th} & Orbital Simulator - Inter IIT Tech Meet, IIT Madras \\
%\end{tabular}
%\sectionsep

%%%%%%%%%%%%%%%%%%%%%%%%%%%%%%%%%%%%%%
%     SOCIETIES
%%%%%%%%%%%%%%%%%%%%%%%%%%%%%%%%%%%%%%

\section{Responsibilities} 
\runsubsection{Co-ordinator}
\descript{| Space Technology and Astronomy Cell}
\location{June 2017 - May 2018 | Mandi, India}
Awarded as Best SnTC Coordinator for 2018-19.
\sectionsep

\end{minipage} 
\end{document}
