%----------------------------------------------------------------------------------------
%	SECTION TITLE
%----------------------------------------------------------------------------------------

\cvsection{Projects}

%----------------------------------------------------------------------------------------
%	SECTION CONTENT
%----------------------------------------------------------------------------------------

\begin{cventries}

%------------------------------------------------

\cventry
{Undergraduate Research} % Job title
{Intelligent Fabric Detection and Classification} % Organization
{IIT Mandi} % Location
{Dec. 2017 - Feb. 2018} % Date(s)
{ % Description(s) of tasks/responsibilities
\begin{cvitems}
\item {Implemented a Deep Learning Model to identify the fabric and classify it into different classes.}
\item {Implemeted it using a Encoder with a CNN classifier, Siamese network for matching. The classification accuracy was 97 \% on the test data}
\item {Can be implemented in major online fabric stores to facilitate the process of picking a fabric without looking at it.}
\item {Made Open Source: https://github.com/shreyasbapat/Fabric-Detection .}
\end{cvitems}
}

%------------------------------------------------

%\cventry
%{Researcher for Detecting the location of camera held at human body using videos} % Job title
%{Undergraduate Research, MANAS Lab(Prof. Aditya Nigam)} % Organization
%{IIT Mandi} % Location
%{Dec. 2017 - Jun. 2018} % Date(s)
%{ % Description(s) of tasks/responsibilities
%\begin{cvitems}
%\item {Classification of Ego-NonEgo Motion}
%\item {Used FlowNet to generate the optical flows for a video and depending on the flow of the pixels, the video was classified.}
%\item {Resnet-50 for classification}
%\end{cvitems} 
%}

%------------------------------------------------

%\cventry
%{Undergraduate Research} % Job title
%{TCM and IRT Generator} % Organization
%{IIT Mandi} % Location
%{Nov. 2017 - Dec. 2017} % Date(s)
%{ % Description(s) of tasks/responsibilities
%\begin{cvitems}
%\item {Created an Autoencoder neural network model on a dataset of original images and TCM, IRT of human palm. As these algorithms take a lot of time, AE is the workaround.}
%\item {Got almost accurate TCM and IRT of Palm images in a very less time.}
%\end{cvitems}
%}

%------------------------------------------------

\cventry
{Open Source Software Developer} % Job title
{Poliastro} % Organization
{India} % Location
{Nov. 2017 - PRESENT} % Date(s)
{ % Description(s) of tasks/responsibilities
\begin{cvitems}
\item {Developed a new plotting class for the plotting module based on plotly. Improved the CI Integrations and developed various methods of plotting trajectories in plotting module.}
\item {Had almost 40\% contributions in the latest major release of poliastro (0.9.0)}
\item {Release Notes v0.9.0: http://docs.poliastro.space/en/v0.9.0/changelog.html#poliastro-0-9-0-2018-04-25}
\item {Project Link: https://github.com/poliastro/poliastro}
\end{cvitems}
}

%------------------------------------------------

\cventry
{Lead Developer} % Job title
{Ayushman Bhava} % Organization
{Design Practicum, IIT Mandi} % Location
{4th Semester - IC201P} % Date(s)
{ % Description(s) of tasks/responsibilities
\begin{cvitems}
\item {Created a super smart medical vending machine with video conferencing with doctor, Payment gateway enabled for online payment.}
\item {Used a Covolutional Neural Network for detecting the cash.}
\item {Used Flask, Tensorflow-Keras, AJAX, HTML, CSS and JS}
\item {Project Link: https://github.com/shreyasbapat/AyushmanBhavaGUI}
\end{cvitems}
}

%------------------------------------------------


\cventry
{Developer} % Job title
{Exoplanet Detection} % Organization
{IIT Mandi} % Location
{Oct. 2017 - Jan. 2018} % Date(s)
{ % Description(s) of tasks/responsibilities
\begin{cvitems}
\item {Created a Machine Learning based prediction model which predicted the presense of Exoplanets on a star by the brightness data of that star over a long period of time.}
\item {Used DFT and Dynamic Time Warping to make the data more readable.}
\item {Project Link: https://github.com/STAC-IITMandi/Exoplanet-Detection}
\end{cvitems}
}

%------------------------------------------------

\cventry
{Developer} % Job title
{Orbital Simulator} % Organization
{IIT Mandi} % Location
{Oct. 2017 - Jan. 2018} % Date(s)
{ % Description(s) of tasks/responsibilities
\begin{cvitems}
\item {Developed a simulator which computed the minimum distance of 500 different asteroids with the planet Mars for the next 5 years.}
\item {This was used to predict the collision of asteroids with Mars.}
\item {Used pyephem, astropy, openpyxl for the problem.}
\item {Project Link: https://github.com/shreyasbapat/Orbital-Simulator}
\end{cvitems}
}

%------------------------------------------------

%\cventry
%{Developer} % Job title
%{Webapp to visualise asteroid trajectories} % Organization
%{India} % Location
%{May 2018 - Aug. 2018} % Date(s)
%{ % Description(s) of tasks/responsibilities
%\begin{cvitems}
%\item {A web app based on dash by potly.}
%\item {Makes the lives of scientist easy. People do not need to know the complicated code base of poliastro. They can do everything using 2-3 clicks on a visual webapp.}
%\end{cvitems}
%}

%------------------------------------------------

\end{cventries}
