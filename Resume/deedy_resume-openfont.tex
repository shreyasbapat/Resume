%%%%%%%%%%%%%%%%%%%%%%%%%%%%%%%%%%%%%%%
% Deedy - One Page Two Column Resume
% LaTeX Template
% Version 1.1 (30/4/2014)
%
% Original author:
% Debarghya Das (http://debarghyadas.com)
%
% Original repository:
% https://github.com/deedydas/Deedy-Resume
%
% IMPORTANT: THIS TEMPLATE NEEDS TO BE COMPILED WITH XeLaTeX
%
% This template uses several fonts not included with Windows/Linux by
% default. If you get compilation errors saying a font is missing, find the line
% on which the font is used and either change it to a font included with your
% operating system or comment the line out to use the default font.
% 
%%%%%%%%%%%%%%%%%%%%%%%%%%%%%%%%%%%%%%
% 
% TODO:
% 1. Integrate biber/bibtex for article citation under publications.
% 2. Figure out a smoother way for the document to flow onto the next page.
% 3. Add styling information for a "Projects/Hacks" section.
% 4. Add location/address information
% 5. Merge OpenFont and MacFonts as a single sty with options.
% 
%%%%%%%%%%%%%%%%%%%%%%%%%%%%%%%%%%%%%%
%
% CHANGELOG:
% v1.1:
% 1. Fixed several compilation bugs with \renewcommand
% 2. Got Open-source fonts (Windows/Linux support)
% 3. Added Last Updated
% 4. Move Title styling into .sty
% 5. Commented .sty file.
%
%%%%%%%%%%%%%%%%%%%%%%%%%%%%%%%%%%%%%%%
%
% Known Issues:
% 1. Overflows onto second page if any column's contents are more than the
% vertical limit
% 2. Hacky space on the first bullet point on the second column.
%
%%%%%%%%%%%%%%%%%%%%%%%%%%%%%%%%%%%%%%

\documentclass[]{deedy-resume-openfont}


\begin{document}

%%%%%%%%%%%%%%%%%%%%%%%%%%%%%%%%%%%%%%
%
%     LAST UPDATED DATE
%
%%%%%%%%%%%%%%%%%%%%%%%%%%%%%%%%%%%%%%
\lastupdated

%%%%%%%%%%%%%%%%%%%%%%%%%%%%%%%%%%%%%%
%
%     TITLE NAME
%
%%%%%%%%%%%%%%%%%%%%%%%%%%%%%%%%%%%%%%


\namesection{Shreyas}{Bapat}{ \urlstyle{same}\url{http://students.iitmandi.ac.in/~b16145/} \\
\href{mailto:b16145@students.iitmandi.ac.in}{b16145@students.iitmandi.ac.in} | \href{mailto:bapat.shreyas@gmail.com}{bapat.shreyas@gmail.com} | (+91) 973 621 0570
}

%%%%%%%%%%%%%%%%%%%%%%%%%%%%%%%%%%%%%%
%
%     COLUMN ONE
%
%%%%%%%%%%%%%%%%%%%%%%%%%%%%%%%%%%%%%%

\begin{minipage}[t]{0.33\textwidth} 

%%%%%%%%%%%%%%%%%%%%%%%%%%%%%%%%%%%%%%
%     EDUCATION
%%%%%%%%%%%%%%%%%%%%%%%%%%%%%%%%%%%%%%

\section{Education} 

\subsection{Indian Institute of Technology Mandi}
\descript{BTech in Electrical Engineering}
\location{Expected Jun 2020 | Mandi, India \\ Cum. GPA: N/A}
\sectionsep


%\descript{BS in Computer Science}
%\location{Expected May 2014 | Ithaca, NY}
%Conc. in Software Engineering \\
%College of Engineering \\
%Dean's List (All Semesters) \\
%\location{ Cum. GPA: 3.92 / 4.0 \\
%Major GPA: 3.94 / 4.0}
%\sectionsep

\subsection{Miss Hill H. Sec. School}
\location{Grad. March 2016 |  Gwalior, India \\ Central Board of Secondary Education \\ Percentage : 93.4\%}
\sectionsep

\subsection{Kendriya Vidyalaya No. 1}
\location{Grad. March 2014 |  Gwalior, India \\ Central Board of Secondary Education \\ CGPA : 10}
\sectionsep
%%%%%%%%%%%%%%%%%%%%%%%%%%%%%%%%%%%%%%
%     LINKS
%%%%%%%%%%%%%%%%%%%%%%%%%%%%%%%%%%%%%%

\section{Links} 
Github:// \href{https://github.com/shreyasbapat}{\custombold{shreyasbapat}} \\
LinkedIn://  \href{https://www.linkedin.com/in/shreyasbapat}{\custombold{shreyasbapat}} \\
\sectionsep

%%%%%%%%%%%%%%%%%%%%%%%%%%%%%%%%%%%%%%
%     COURSEWORK
%%%%%%%%%%%%%%%%%%%%%%%%%%%%%%%%%%%%%%

\section{Coursework}

\subsection{Undergraduate}
Computation for Engineers \\
Programming and Data Structure Practicum \\
{\footnotesize \textit{\textbf{(Teaching Asst) }}} \\
Data Structures and Algorithm \\
Computer Organisation \\
Device Electronics \\
Signals and Systems \\
Design Practicum \\
Deep Learning \\
Applied Electronics \\
Network Theory \\
\sectionsep

%%%%%%%%%%%%%%%%%%%%%%%%%%%%%%%%%%%%%%
%     SKILLS
%%%%%%%%%%%%%%%%%%%%%%%%%%%%%%%%%%%%%%

\section{Skills}
\subsection{Programming}
\location{Intermediate:}
Python \textbullet{}   Shell \textbullet{} C \textbullet{} C++ \\
HTML5 \textbullet{} \LaTeX\ \textbullet{} CSS3 \textbullet{} Ubuntu\\ 
\location{Beginner:}
JavaScript \textbullet{} PHP \textbullet{} MySQL \textbullet{} Assembly \\
\sectionsep

\section{Interests}
Deep Learning\\
Machine Learning\\
Open Source Development\\
Astrodynamics\\
Astrophysics\\
Astronomy


%%%%%%%%%%%%%%%%%%%%%%%%%%%%%%%%%%%%%%
%
%     COLUMN TWO
%
%%%%%%%%%%%%%%%%%%%%%%%%%%%%%%%%%%%%%%

\end{minipage} 
\hfill
\begin{minipage}[t]{0.66\textwidth} 

%%%%%%%%%%%%%%%%%%%%%%%%%%%%%%%%%%%%%%
%     EXPERIENCE
%%%%%%%%%%%%%%%%%%%%%%%%%%%%%%%%%%%%%%

\section{Experience}

\runsubsection{Teaching Assistant}
\descript{| IC 250 - Programming and Data Structure Practicum }
\location{Feb 2018 - June 2018 | Dr. Padmanabhan Rajan}
\vspace{\topsep} % Hacky fix for awkward extra vertical space

\sectionsep

%\runsubsection{Google}
%\descript{| Software Engineering Intern }
%\location{May 2013 – Aug 2013 | Mountain View, CA}
%\begin{tightemize}
%\item Worked on the YouTube Captions team in primarily vanilla Javascript and Python to plan, design and develop the full stack implementation of a new framework to add and edit Automatic Speech Recognition captions.\item Created a backbone.js-like framework for the Captions editor.\item All code was reviewed, perfected, and pushed to production.\end{tightemize}
%\sectionsep

%\runsubsection{Phabricator}
%\descript{| Open Source Contributor \& Team Leader}
%\location{Jan 2013 – May 2013 | Palo Alto, CA \& Ithaca, NY}
%\begin{tightemize}
%\item Phabricator is used daily by Facebook, Dropbox, Quora, Asana and more.\item I created the Meme generator, the entire Lipsum application, ported Tokens to different apps, fixed many bugs and more in PHP and Shell.\item Led a team from MIT, Cornell, IC London and UHelsinki for the project.\end{tightemize}
%\sectionsep

%%%%%%%%%%%%%%%%%%%%%%%%%%%%%%%%%%%%%%
%     RESEARCH
%%%%%%%%%%%%%%%%%%%%%%%%%%%%%%%%%%%%%%

\section{Projects}
\runsubsection{Poliastro}
\descript{| Open Source Contributor}
\location{Jan 2018 – Present | IIT Mandi, India}
Contributed to a Open Source Python library, \textbf{\href{http://github.com/poliastro/poliastro}{Poliastro}} - a library for orbital mechanics and related computations.
\sectionsep

\runsubsection{Fabric Detection}
\descript{| Undergraduate Research}
\location{Jan 2018 – Present | IIT Mandi, India}
Worked with \textbf{\href{http://faculty.iitmandi.ac.in/~aditya/}{Dr. Aditya Nigam}} to create \textbf{Fabric Detection}, a deep learning approach to classify and match different fabrics using their microscopic images. 
\sectionsep

\runsubsection{Orbital Simulator}
\descript{| Developer}
\location{Jan 2018 | IIT Madras, India}
This Orbital Simulator was made as a part of Inter IIT Tech Meet. It computed the nearest distance of Asteriods to Mars for the period of 5 Years. 
\sectionsep


\runsubsection{Projecthiko 1.O}
\descript{| Developer}
\location{Feb 2017 – May 2017 | IIT Mandi, India}
Lead the development of \textbf{Projecthiko 1.O - The Voice and Internet Enabled Smart Home}, the cheapest home automation system. It was presented i n Aavishkar 2k17 as a project for course IC161P- Applied Electronics Lab.
\sectionsep



\section{Publications}

\runsubsection{Projecthiko 1.O - The Voice and Internet Enable Smart Home}
\descript{| Co-Author}
\location{June 2017 | IJETSR ISSN: 2394-3386}
Aggarwal, V.,\textbf{Bapat, S.}, Kadela, N., Kumar, P., Matta, R., Yadav, N.
\sectionsep


%%%%%%%%%%%%%%%%%%%%%%%%%%%%%%%%%%%%%%
%     AWARDS
%%%%%%%%%%%%%%%%%%%%%%%%%%%%%%%%%%%%%%

\section{Awards} 
\begin{tabular}{rll}
2013	     & 5th/7500  & Aryabhat Astronomy Quiz\\
2014	     & 2nd/8200  & Aryabhat Astronomy Quiz\\
2017	     & 1\textsuperscript{st}  & Technex-Exploring the Interstellar, IIT BHU\\
2017	     & 4\textsuperscript{th}  & Eyes on the Sky - Inter IIT Tech Meet, IIT Kanpur\\
2017     & Presented a paper & International Conference ICETSMI at IETE, Delhi  \\
2018     & 5\textsuperscript{th} & Orbital Simulator - Inter IIT Tech Meet, IIT Madras \\
\end{tabular}
\sectionsep

%%%%%%%%%%%%%%%%%%%%%%%%%%%%%%%%%%%%%%
%     SOCIETIES
%%%%%%%%%%%%%%%%%%%%%%%%%%%%%%%%%%%%%%

\section{Responsibilities} 
\runsubsection{Co-ordinator}
\descript{| Space Technology and Astronomy Cell}
\location{June 2017 - Present | IIT Mandi, India}
\sectionsep

\runsubsection{Core Member}
\descript{| Robotronics Club}
\location{June 2017 - Present | IIT Mandi, India}
\sectionsep

\runsubsection{Event Co-ordinator}
\descript{| Strip-Mics, Exodia}
\location{April 2017 | IIT Mandi, India}
\sectionsep

\sectionsep

\end{minipage} 
\end{document}